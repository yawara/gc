\documentclass{jarticle}

\setlength{\voffset}{-65pt}
\setlength{\oddsidemargin}{-5mm}
\setlength{\textwidth}{490pt}
\setlength{\textheight}{700pt}

\usepackage{graphicx}
\usepackage{amsmath,amssymb,pifont,colortbl,amscd, wrapfig, ascmac}
\usepackage{amsthm}
\usepackage{url}
\usepackage{bm}
\newtheorem{theorem}{定理}
\newtheorem{definition}{定義}
\newtheorem{example}{例}

\def\ev{\mathrm{ev}}
\def\c{\mathop{\mathrm{cts}}\nolimits}
\def\odd{\mathop{\mathrm{odd}}\nolimits}
\def\diag{\mathop{\mathrm{diag}}\nolimits}
\def\mod{\mathop{\mathrm{mod}}\nolimits}
\def\deg{\mathop{\mathrm{deg}}\nolimits}
\def\inv{\mathop{\mathrm{Inv}}\nolimits}
\def\cl{\mathop{\mathrm{cl}}\nolimits}
\def\ad{\mathop{\mathrm{ad}}\nolimits}
\newcommand{\sad}{\overline{\ad}}
\def\tr{\mathop{\mathrm{tr}}\nolimits}
\def\End{\mathop{\mathrm{End}}\nolimits}
\def\id{\mathop{\mathrm{id}}\nolimits}
\def\ev{\mathop{\mathrm{ev}}\nolimits}
\def\coev{\mathop{\mathrm{coev}}\nolimits}
\def\coad{\mathop{\mathrm{coad}}\nolimits}
\def\Ob{\mathop{\mathrm{Ob}}\nolimits}
\def\Hom{\mathop{\mathrm{Hom}}\nolimits}
\def\im{\mathop{\mathrm{Im}}\nolimits}
\def\Span{\mathop{\mathrm{Span}}\nolimits}
\def\ideal{\mathop{\mathrm{ideal}}\nolimits}
\def\co{\colon\thinspace}
%for U_h
\newcommand{\uqenh}[1]{ (\bar U_q^{\ev})\,\hat  {}^{\;\hat  \otimes #1}}
\newcommand{\uqen}[1]{ (\bar U_q^{\ev})\;\tilde {}^{\;\tilde \otimes #1}}
\newcommand{\uqe}{\bar U_q^{\ev}}
\newcommand{\uqz}{\bar U_q^0}
\newcommand{\uqze}{\bar U_q^{\ev 0}}
\newcommand{\uq}{\bar U_q}
\newcommand{\muq}{\mathcal{ U}_q}
\newcommand{\muqe}{\mathcal{ U}_q^{\ev}}
\newcommand{\uqzq}{U_{\mathbb{Z},q}}
\newcommand{\uqzqe}{(U_{\mathbb{Z},q})^{\ev}}
\newcommand{\uqn}[1]{\bar U_q^{\otimes {#1}}}
\newcommand{\uhn}[1]{\bar U_h^{\hat \otimes {#1}}}
\newcommand{\f}[1]{\tilde F^{({#1})}}
\newcommand{\e}[1]{\tilde E^{({#1})}}
\newcommand{\Z}{\mathbb{Z}[q,q^{-1}]}
\def\ZA{\mathbb{P}^{\mathrm{fin}}\big( \Hom_{\mathcal{A}}(0,g)\big)}

\def\deg{\mathop{\mathrm{deg}}\nolimits}
\def\inv{\mathop{\mathrm{Inv}}\nolimits}
\def\cl{\mathop{\mathrm{cl}}\nolimits}
\def\ad{\mathop{\mathrm{ad}}\nolimits}
\def\tr{\mathop{\mathrm{tr}}\nolimits}
\def\End{\mathop{\mathrm{End}}\nolimits}
\def\id{\mathop{\mathrm{id}}\nolimits}
\def\ev{\mathop{\mathrm{ev}}\nolimits}
\def\coev{\mathop{\mathrm{coev}}\nolimits}
\def\coad{\mathop{\mathrm{coad}}\nolimits}
\def\Ob{\mathop{\mathrm{Ob}}\nolimits}
\def\Hom{\mathop{\mathrm{Hom}}\nolimits}
\def\Sets{\mathop{\mathrm{Sets}}\nolimits}
\def\im{\mathop{\mathrm{Im}}\nolimits}
\def\co{\colon\thinspace}

%%% ywr extend 
\def\d{\mathrm d}
\def\grad{\mathrm grad}
\def\rot{\mathrm{\bm rot}}
\def\div{\mathrm{div}}
\def\e{\mathrm{e}}
%%%

\begin{document}

\title{微分積分続論(ベクトル解析)} 
\author{鈴木 咲衣}
\date{平成27年度前期}
\maketitle

\begin{center} {\Large 演習問題6 } \end{center}
  \begin{enumerate}
  \item  \cite[問題4.1]{koba}次の面積分を求めよ. \label{41}
  \begin{enumerate}
  \item $\int_{0}^{1}\int_{0}^{1} (x+y)^{2}dxdy $
  \item $ \int_{0}^{2}\int_{0}^{1} e^{x-y}dxdy $
  \item $\int_{0}^{\frac{\pi}{2}} \int_{0}^{\frac{\pi}{2}} \cos(x+2y) dxdy $
  \end{enumerate}
  \item  \cite[問題4.2]{koba} 問\ref{41}の重積分の順序を入れ替えて計算せよ.
  \item  \cite[問題4.5]{koba} 領域$D$を原点を中心とした半径$1$の円とする.次の面積分を極座標を用いて求めよ.
  \begin{enumerate}
  \item $\int_{D} x^{2}dS$
  \item $\int_{D} (x^{2}+y^{2)}dS$
  \item$ \int_{D} x^{2}ydS$
  \item$ \int_{D} e^{x^{2}+y^{2}}dS$
  \end{enumerate}
  \end{enumerate}
\newpage

\begin{center} {\Large 演習問題6 解答} \end{center}
  \begin{enumerate}
    \item
    \begin{enumerate}
      \item
        \begin{eqnarray*} 
          \int_0^1 \int_0^1 (x+y)^2 \d x \d y & = & \int_0^1 \frac{1}{3} \left[(x+y)^3 \right]_0^1 \d y \\
          & = & \frac{1}{3} \int_0^1 \left\{ (1+y)^3 - y^3 \right\} \d y \\
          & = & \frac{1}{3} \cdot \frac{1}{4} \left[ (y+1)^4 - y^4 \right]_0^1 \\
          & = & \frac{1}{12} \cdot \{ (2^4-1)-1 \} = \frac{7}{6}
        \end{eqnarray*}
      \item
        \begin{eqnarray*}
          \int_0^2 \int_0^1 \e^{x-y} \d x \d y & = & \int_0^2 \e^{-y} \d y \int_0^1 \e^{x} \d x \\
          & = & \left[ - \e^{-y} \right]_0^2 \left[ \e^{x} \right]_0^1 \\
          & = & (1 - \e^{-2} )(\e -1 )
        \end{eqnarray*}
      \item
        \begin{eqnarray*}
          \int_0^{\frac{\pi}{2}} \int_0^{\frac{\pi}{2}} \cos(x+2y) \d x \d y & = & \int_0^{\frac{\pi}{2}} \left[ \sin(x+2y) \right]_0^{\frac{\pi}{2}} \d y \\
          & = & \int_0^{\frac{\pi}{2}} \{ \sin(2y+\frac{\pi}{2}) - \sin(2y) \} \d y =  \int_0^{\frac{\pi}{2}} \{ \cos(2y) - \sin(2y) \} \d y \\
          & = & \int_0^{\frac{\pi}{2}} \{ - \sin(2y) \} \d y = \frac{1}{2} \left[ \cos(2y) \right]_0^{\frac{\pi}{2}} = -1
        \end{eqnarray*}
    \end{enumerate}
    \item (省力する)
    \item
      \begin{enumerate}
        \item
          \begin{eqnarray*}
            \int_{D} x^2 \d S & = & \int_0^1 r^3 \d r \int_0^{2\pi} \cos^2 \theta \d \theta \\
            & = & \frac{1}{4} \int_0^{2\pi} \frac{1+\cos{2\theta}}{2} \d \theta \\
            & = & \frac{1}{4} \cdot \pi = \frac{\pi}{4}
          \end{eqnarray*}
        \item
          \begin{eqnarray*}
            \int_{D} (x^2+y^2) \d S & = & \int_0^1 r^3 \d r \int_0^{2\pi} \d \theta \\
            & = & \frac{1}{4} \cdot 2\pi = \frac{\pi}{2}
          \end{eqnarray*}
        \item
          \begin{eqnarray*}
            \int_{D} x^2y \d S & = & \int_0^1 r^4 \d r \int_0^{2\pi} \cos^2 \theta \sin \theta \d \theta \\
            & = & \frac{1}{5} \left[ -\frac{1}{3} \cos^3 \theta \right]_0^{2\pi} = 0
          \end{eqnarray*}
        \item
          \begin{eqnarray*}
            \int_{D} \e^{x^2+y^2} \d S & = & \int_0^1 r \e^{r^2} \d r \int_0^{2\pi} \d \theta \\
            & = & 2 \pi \left[ \frac{1}{2} \e^{r^2} \right]_0^1 = \pi ( \e - 1 )
          \end{eqnarray*}
      \end{enumerate}
  \end{enumerate}
  
\newpage

\begin{thebibliography}{99}
\bibitem{koba} 小林亮,高橋大輔「ベクトル解析入門」(東京大学出版会)
\end{thebibliography}

\end{document}