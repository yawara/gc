\documentclass{jarticle}

\setlength{\voffset}{-65pt}
\setlength{\oddsidemargin}{-5mm}
\setlength{\textwidth}{490pt}
\setlength{\textheight}{700pt}

\usepackage{graphicx}
\usepackage{amsmath,amssymb,pifont,colortbl,amscd, wrapfig, ascmac}
\usepackage{amsthm}
\usepackage{url}
\usepackage{bm}
\newtheorem{theorem}{定理}
\newtheorem{definition}{定義}
\newtheorem{example}{例}

\def\ev{\mathrm{ev}}
\def\c{\mathop{\mathrm{cts}}\nolimits}
\def\odd{\mathop{\mathrm{odd}}\nolimits}
\def\diag{\mathop{\mathrm{diag}}\nolimits}
\def\mod{\mathop{\mathrm{mod}}\nolimits}
\def\deg{\mathop{\mathrm{deg}}\nolimits}
\def\inv{\mathop{\mathrm{Inv}}\nolimits}
\def\cl{\mathop{\mathrm{cl}}\nolimits}
\def\ad{\mathop{\mathrm{ad}}\nolimits}
\newcommand{\sad}{\overline{\ad}}
\def\tr{\mathop{\mathrm{tr}}\nolimits}
\def\End{\mathop{\mathrm{End}}\nolimits}
\def\id{\mathop{\mathrm{id}}\nolimits}
\def\ev{\mathop{\mathrm{ev}}\nolimits}
\def\coev{\mathop{\mathrm{coev}}\nolimits}
\def\coad{\mathop{\mathrm{coad}}\nolimits}
\def\Ob{\mathop{\mathrm{Ob}}\nolimits}
\def\Hom{\mathop{\mathrm{Hom}}\nolimits}
\def\im{\mathop{\mathrm{Im}}\nolimits}
\def\Span{\mathop{\mathrm{Span}}\nolimits}
\def\ideal{\mathop{\mathrm{ideal}}\nolimits}
\def\co{\colon\thinspace}
%for U_h
\newcommand{\uqenh}[1]{ (\bar U_q^{\ev})\,\hat  {}^{\;\hat  \otimes #1}}
\newcommand{\uqen}[1]{ (\bar U_q^{\ev})\;\tilde {}^{\;\tilde \otimes #1}}
\newcommand{\uqe}{\bar U_q^{\ev}}
\newcommand{\uqz}{\bar U_q^0}
\newcommand{\uqze}{\bar U_q^{\ev 0}}
\newcommand{\uq}{\bar U_q}
\newcommand{\muq}{\mathcal{ U}_q}
\newcommand{\muqe}{\mathcal{ U}_q^{\ev}}
\newcommand{\uqzq}{U_{\mathbb{Z},q}}
\newcommand{\uqzqe}{(U_{\mathbb{Z},q})^{\ev}}
\newcommand{\uqn}[1]{\bar U_q^{\otimes {#1}}}
\newcommand{\uhn}[1]{\bar U_h^{\hat \otimes {#1}}}
\newcommand{\f}[1]{\tilde F^{({#1})}}
\newcommand{\e}[1]{\tilde E^{({#1})}}
\newcommand{\Z}{\mathbb{Z}[q,q^{-1}]}
\def\ZA{\mathbb{P}^{\mathrm{fin}}\big( \Hom_{\mathcal{A}}(0,g)\big)}

\def\deg{\mathop{\mathrm{deg}}\nolimits}
\def\inv{\mathop{\mathrm{Inv}}\nolimits}
\def\cl{\mathop{\mathrm{cl}}\nolimits}
\def\ad{\mathop{\mathrm{ad}}\nolimits}
\def\tr{\mathop{\mathrm{tr}}\nolimits}
\def\End{\mathop{\mathrm{End}}\nolimits}
\def\id{\mathop{\mathrm{id}}\nolimits}
\def\ev{\mathop{\mathrm{ev}}\nolimits}
\def\coev{\mathop{\mathrm{coev}}\nolimits}
\def\coad{\mathop{\mathrm{coad}}\nolimits}
\def\Ob{\mathop{\mathrm{Ob}}\nolimits}
\def\Hom{\mathop{\mathrm{Hom}}\nolimits}
\def\Sets{\mathop{\mathrm{Sets}}\nolimits}
\def\im{\mathop{\mathrm{Im}}\nolimits}
\def\co{\colon\thinspace}

%%% ywr extend 
\def\d{\mathrm d}
\def\grad{\mathrm grad}
\def\rot{\mathrm{\bm rot}}
\def\div{\mathrm{div}}
%%%

\begin{document}

\title{微分積分続論(ベクトル解析)} 
\author{鈴木 咲衣}
\date{平成27年度前期}
\maketitle

\begin{center} {\Large 演習問題10 } \end{center}
\begin{enumerate}
\item \cite[問題5.2]{koba}
以下の立体をパラメータで表示し,その体積を求めよ.
\begin{enumerate}
\item 点$(1,2,3)$を中心とする半径$1$および半径$2$の球面にはさまれた部分.
\item $z$軸を中心軸とする半径$1$および$2$の円筒にはさまれた領域の$0\leq z\leq 1$の部分.
\end{enumerate} 
\item \cite[問題5.5]{koba}
原点を中心とする半径$1$の球の内部領域を$\Omega$とする.このとき次の体積分を求めよ.
\begin{enumerate}
\item $\int_{\Omega} (x^{2}+y^{2}+z^{2}) dV$
\item $\int_{\Omega} z^{2} dV$
\end{enumerate} 
\item \cite[問題5.6]{koba}
$z$軸を中心軸とする半径$2$の円筒内部の$0\leq z\leq 1$の領域を$\Omega$とする.このとき次の体積分を求めよ.
\begin{enumerate}
\item $\int_{\Omega} 1+xdV$
\item $\int_{\Omega} zdV$
\item $\int_{\Omega} (x^{2}+y^{2})dV$
\end{enumerate} 
\item  \cite[問題8.10]{koba}
領域$D$のパラメータを$\psi$としたとき,$D$の体積は
$V=\frac{1}{3}\int_{\partial D} \psi \cdot d\bm S$となることを示せ.
\item \cite[章末問題8.4]{koba}パラメータ$\psi$で表される$3$次元空間の領域$D$に対して次の等式が成り立つことを示せ.
\begin{align*}
\int_{\partial D} \frac{1}{r^{3}} \psi \cdot d \bm S =\begin{cases} 0, \quad (0,0,0) \not \in D,
\\
4\pi, \quad  (0,0,0) \in D.
\end{cases}
\end{align*}
\end{enumerate}
\newpage

\begin{center} {\Large 演習問題10 解答} \end{center}
  \begin{enumerate}
    \item
      \begin{enumerate}
        \item
          次のように極座標でパラメーター表示しよう(当然、パラメーター表示は複数考えられる)。
          \[ (x,y,z)=(1,2,3)+r(\cos\varphi\sin\theta,\sin\varphi\sin\theta,\cos\theta) \]
          ただし各パラメーターは$D: 0 \leq \varphi \leq 2\pi, 0 \leq \theta \leq \pi, 1 \leq r \leq 2$の範囲内で動くものとする。 
          \begin{eqnarray*}
            \int_D \d V & = & \int_D r^2 \sin\theta \d \theta \d \varphi \d r \\
            & = & \int_1^2 r^2 \d r \int_0^{2\pi} \d \varphi \int_0^{\pi} \sin \theta \d \theta \\
            & = & 4\pi \left[ \frac{r^3}{3} \right]_1^2 = 4\pi \cdot \frac{7}{3} = \frac{28}{3} \pi
          \end{eqnarray*}
        \item
          次のように円柱座標でパラメーター表示しよう(当然、パラメーター表示は複数考えられる)。
          \[ (x,y,z)=(r\cos\phi,r\sin\phi,z) \]
          ただし各パラメーターは$D: 1 \leq r \leq 2, 0 \leq \phi < 2\pi, 0 \leq z \leq 1$の範囲内で動くものとする。
          \begin{eqnarray*}
            \int_D \d V & = & \int_D r \d r \d \phi \d z \\
            & = & \int_1^2 r \d r \int_0^{2\pi} \d \phi \int_0^1 \d z \\
            & = & 2\pi \left[ \frac{r^2}{2} \right]_1^2 = 2\pi \cdot \frac{3}{2} = 3\pi
          \end{eqnarray*}
      \end{enumerate}
    \item
      \begin{enumerate}
        \item
          \begin{eqnarray*}
            \int_\Omega (x^2+y^2+z^2) \d V & = & \int_\Omega r^4 \sin\theta \d r \d \theta \d \varphi \\
            & = & \int_0^1 r^4 \d r \int_0^{\pi} \sin \theta \d \theta \int_0^{2\pi} \d \varphi \\
            & = & \frac{4}{5} \pi
          \end{eqnarray*}
        \item
          \begin{eqnarray*}
            \int_\Omega z^2 \d V & = & \int_\Omega r^4 \cos^2 \theta \sin \theta \d r \d \theta \d \varphi \\
            & = & \int_0^1 r^4 \d r \int_0^{\pi} \cos^2 \theta \sin \theta \d \theta \int_0^{2\pi} \d \varphi \\
            & = & \frac{2}{5} \pi \int_{-1}^{1} u^2 \d u \\
            & = & \frac{2}{5} \pi \cdot \frac{2}{3} = \frac{4}{15} \pi
          \end{eqnarray*}
      \end{enumerate}
    \item
      \begin{enumerate}
        \item
          \begin{eqnarray*}
            \int_\Omega (1+x) \d V & = & \int_\Omega (1+r \cos \phi) r \d r \d \phi \d z \\
            & = & \int_0^2 r \d r \int_0^{2\pi} \d \phi \int_0^1 \d z \\
            & = & \frac{4}{2} \cdot 2\pi \cdot 1 = 4 \pi
          \end{eqnarray*}
        \item
          \begin{eqnarray*}
            \int_\Omega z \d V & = & \int_\Omega z r \d r \d \phi \d z \\
            & = & \int_0^2 \int_0^{2\pi} \d \phi \int_0^1 z \d z \\
            & = & \frac{4}{2} \cdot 2\pi \cdot \frac{1}{2} = 2\pi
          \end{eqnarray*}
        \item
          \begin{eqnarray*}
            \int_\Omega (x^2+y^2) \d V & = & \int_\Omega r^3 \d r \d \phi \d z \\
            & = & \int_0^2 r^3 \d r \int_0^{2\pi} \d \phi \int_0^1 \d z \\
            & = & \frac{16}{4} \cdot 2\pi \cdot 1 = 8 \pi
          \end{eqnarray*}
      \end{enumerate}
    \item
    \item
  \end{enumerate}
\end{document}