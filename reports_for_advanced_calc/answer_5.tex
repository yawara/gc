\documentclass{jarticle}

\setlength{\voffset}{-65pt}
\setlength{\oddsidemargin}{-5mm}
\setlength{\textwidth}{490pt}
\setlength{\textheight}{700pt}

\usepackage{graphicx}
\usepackage{amsmath,amssymb,pifont,colortbl,amscd, wrapfig, ascmac}
\usepackage{amsthm}
\usepackage{url}
\usepackage{bm}
\newtheorem{theorem}{定理}
\newtheorem{definition}{定義}
\newtheorem{example}{例}

\def\ev{\mathrm{ev}}
\def\c{\mathop{\mathrm{cts}}\nolimits}
\def\odd{\mathop{\mathrm{odd}}\nolimits}
\def\diag{\mathop{\mathrm{diag}}\nolimits}
\def\mod{\mathop{\mathrm{mod}}\nolimits}
\def\deg{\mathop{\mathrm{deg}}\nolimits}
\def\inv{\mathop{\mathrm{Inv}}\nolimits}
\def\cl{\mathop{\mathrm{cl}}\nolimits}
\def\ad{\mathop{\mathrm{ad}}\nolimits}
\newcommand{\sad}{\overline{\ad}}
\def\tr{\mathop{\mathrm{tr}}\nolimits}
\def\End{\mathop{\mathrm{End}}\nolimits}
\def\id{\mathop{\mathrm{id}}\nolimits}
\def\ev{\mathop{\mathrm{ev}}\nolimits}
\def\coev{\mathop{\mathrm{coev}}\nolimits}
\def\coad{\mathop{\mathrm{coad}}\nolimits}
\def\Ob{\mathop{\mathrm{Ob}}\nolimits}
\def\Hom{\mathop{\mathrm{Hom}}\nolimits}
\def\im{\mathop{\mathrm{Im}}\nolimits}
\def\Span{\mathop{\mathrm{Span}}\nolimits}
\def\ideal{\mathop{\mathrm{ideal}}\nolimits}
\def\co{\colon\thinspace}
%for U_h
\newcommand{\uqenh}[1]{ (\bar U_q^{\ev})\,\hat  {}^{\;\hat  \otimes #1}}
\newcommand{\uqen}[1]{ (\bar U_q^{\ev})\;\tilde {}^{\;\tilde \otimes #1}}
\newcommand{\uqe}{\bar U_q^{\ev}}
\newcommand{\uqz}{\bar U_q^0}
\newcommand{\uqze}{\bar U_q^{\ev 0}}
\newcommand{\uq}{\bar U_q}
\newcommand{\muq}{\mathcal{ U}_q}
\newcommand{\muqe}{\mathcal{ U}_q^{\ev}}
\newcommand{\uqzq}{U_{\mathbb{Z},q}}
\newcommand{\uqzqe}{(U_{\mathbb{Z},q})^{\ev}}
\newcommand{\uqn}[1]{\bar U_q^{\otimes {#1}}}
\newcommand{\uhn}[1]{\bar U_h^{\hat \otimes {#1}}}
\newcommand{\f}[1]{\tilde F^{({#1})}}
\newcommand{\e}[1]{\tilde E^{({#1})}}
\newcommand{\Z}{\mathbb{Z}[q,q^{-1}]}
\def\ZA{\mathbb{P}^{\mathrm{fin}}\big( \Hom_{\mathcal{A}}(0,g)\big)}

\def\deg{\mathop{\mathrm{deg}}\nolimits}
\def\inv{\mathop{\mathrm{Inv}}\nolimits}
\def\cl{\mathop{\mathrm{cl}}\nolimits}
\def\ad{\mathop{\mathrm{ad}}\nolimits}
\def\tr{\mathop{\mathrm{tr}}\nolimits}
\def\End{\mathop{\mathrm{End}}\nolimits}
\def\id{\mathop{\mathrm{id}}\nolimits}
\def\ev{\mathop{\mathrm{ev}}\nolimits}
\def\coev{\mathop{\mathrm{coev}}\nolimits}
\def\coad{\mathop{\mathrm{coad}}\nolimits}
\def\Ob{\mathop{\mathrm{Ob}}\nolimits}
\def\Hom{\mathop{\mathrm{Hom}}\nolimits}
\def\Sets{\mathop{\mathrm{Sets}}\nolimits}
\def\im{\mathop{\mathrm{Im}}\nolimits}
\def\co{\colon\thinspace}

%%% ywr extend 
\def\d{\mathrm d}
\def\grad{\mathrm grad}
\def\rot{\mathrm{\bm rot}}
\def\div{\mathrm{div}}
%%%

\begin{document}

\title{微分積分続論(ベクトル解析)} 
\author{鈴木 咲衣}
\date{平成27年度前期}
\maketitle

\begin{center} {\Large 演習問題5 } \end{center}

\begin{enumerate}
\item 次のベクトル場$\bm V$の単位円周$C: \bm r(t)=(\cos t, \sin t)$, $0\leq t\leq 2\pi$,に沿った渦巻き量と湧き出し量を求めよ.
\begin{enumerate}
\item
$\bm V=\bm e_{r}$
\item
$\bm V=\bm e_{\theta}$
\end{enumerate}
\item 5章(14)--(16)を示せ.
\item 5章(33)--(35)を示せ.
\item \cite[問題6.16]{koba} 次の各ベクトル場$\bm V$について,回転$\mathrm {rot} V(x,y)$と発散$\mathrm {div} V(x,y)$を求めよ.
\begin{enumerate}
\item $\bm V=(1,0)$
\item $\bm V=(1,1)$
\item $\bm V=(y,0)$
\item $\bm V=(0,x)$
\item $\bm V=(x,0)$
\item $\bm V=(0,-y)$
\item $\bm V=(x,y)$
\item $\bm V=(x,-y)$
\item $\bm V=(-y,x)$
\item $\bm V=(y,x)$
\end{enumerate}
\item \cite[問題6.24, 6.28]{koba}ベクトル場$\bm V$が極座標表示で$\bm V(r, \theta)=P(r,\theta) \bm e_{r}+ Q (r, \theta) \bm e_{\theta}$と表されているとき,
\begin{enumerate}
\item $\mathrm {rot} \bm V (r,\theta)$を計算せよ.
\item $\mathrm {div} \bm V (r,\theta)$を計算せよ. 
\end{enumerate}
\end{enumerate}

\newpage

\begin{center} {\Large 演習問題5 解答} \end{center}
  \begin{enumerate}
    \item
      \begin{enumerate}
        \item
          渦巻き量は$0$,湧き出し量は$2\pi$
        \item
          渦巻き量は$2\pi$,湧き出し量は$0$

      \end{enumerate}
    \item
    \begin{align}
\int_{C_{2}}\bm V \cdot \bm t ds&=\int_{-\frac{h}{2}}^{\frac{h}{2}}(P(x+\frac{h}{2}, y+s), Q(x+\frac{h}{2}, y+s)) \cdot (0,1) ds
\\
&=\int_{-\frac{h}{2}}^{\frac{h}{2}}Q(x+\frac{h}{2}, y+s)ds
\end{align}
    \begin{align}
\int_{C_{3}}\bm V \cdot \bm t ds&=\int_{-\frac{h}{2}}^{\frac{h}{2}}(P(x+s, y+\frac{h}{2}), Q(x+s, y+\frac{h}{2})) \cdot (-1,0) ds
\\
&=\int_{-\frac{h}{2}}^{\frac{h}{2}}-P(x+s, y+\frac{h}{2})ds
\end{align}
    \begin{align}
\int_{C_{4}}\bm V \cdot \bm t ds&=\int_{-\frac{h}{2}}^{\frac{h}{2}}(P(x-\frac{h}{2}, y+s), Q(x-\frac{h}{2}, y+s) \cdot (0,-1) ds
\\
&=\int_{-\frac{h}{2}}^{\frac{h}{2}}-Q(x-\frac{h}{2}, y+s)ds
\end{align}
    \item
    \begin{align}
\int_{C_{2}}\bm V \cdot \bm n ds&=\int_{-\frac{h}{2}}^{\frac{h}{2}}(P(x+\frac{h}{2}, y+s), Q(x+\frac{h}{2}, y+s)) \cdot (1,0) ds
\\
&=\int_{-\frac{h}{2}}^{\frac{h}{2}}P(x+\frac{h}{2}, y+s)ds
\end{align}
    \begin{align}
\int_{C_{3}}\bm V \cdot \bm n ds&=\int_{-\frac{h}{2}}^{\frac{h}{2}}(P(x+s, y+\frac{h}{2}), Q(x+s, y+\frac{h}{2})) \cdot (0,1) ds
\\
&=\int_{-\frac{h}{2}}^{\frac{h}{2}}Q(x+s, y+\frac{h}{2})ds
\end{align}
    \begin{align}
\int_{C_{4}}\bm V \cdot \bm n ds&=\int_{-\frac{h}{2}}^{\frac{h}{2}}(P(x-\frac{h}{2}, y+s), Q(x-\frac{h}{2}, y+s)) \cdot (-1,0) ds
\\
&=\int_{-\frac{h}{2}}^{\frac{h}{2}}-P(x-\frac{h}{2}, y+s)ds
\end{align}
  
    \item
      \begin{enumerate}
        \item 
          \[ \div \bm V = 0, \quad  \rot \bm V = 0 \]
        \item
          \[ \div \bm V = 0, \quad \rot \bm V = 0 \]
        
        \item
          \[ \div \bm V = 0, \quad \rot \bm V = -1 \]
        \item
          \[ \div \bm V = 0, \quad \rot \bm V = 1 \]
        \item 
          \[ \div \bm V = 1, \quad \rot \bm V = 0 \]
        \item
          \[ \div \bm V = -1, \quad \rot \bm V = 0 \]
        
        \item
          \[ \div \bm V = 2, \quad \rot \bm V = 0 \]
        \item
          \[ \div \bm V = 0, \quad \rot \bm V = 0 \]
      
        \item
          \[ \div \bm V = 0, \quad \rot \bm V = 2 \]
        \item
          \[ \div \bm V = 0, \quad \rot \bm V = 0 \]
      \end{enumerate}
    \item
      \begin{enumerate}
   \item \begin{align*}
   \mathrm {rot} \bm V (r,\theta)&= \frac{\partial Q}{\partial r} +\frac{1}{r}Q-\frac{1}{r} \frac{\partial P}{\partial \theta}
   \\
   &= \frac{1}{r} \frac{\partial }{\partial r}(rQ)-\frac{1}{r} \frac{\partial P}{\partial \theta}
   \end{align*}
\item 
\begin{align*} 
  \mathrm {div} \bm V (r,\theta)&= \frac{\partial P}{\partial r} +\frac{1}{r}P+\frac{1}{r} \frac{\partial Q}{\partial \theta}
   \\
   &= \frac{1}{r} \frac{\partial }{\partial r}(rP)+\frac{1}{r} \frac{\partial Q}{\partial \theta}
   \end{align*}
          
      \end{enumerate}
  \end{enumerate}
  

\newpage

\begin{thebibliography}{99}
\bibitem[2]{koba} 小林亮,高橋大輔「ベクトル解析入門」(東京大学出版会)
\end{thebibliography}

\end{document}