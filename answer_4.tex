\documentclass{jarticle}

\setlength{\voffset}{-65pt}
\setlength{\oddsidemargin}{-5mm}
\setlength{\textwidth}{490pt}
\setlength{\textheight}{700pt}

\usepackage{graphicx}
\usepackage{amsmath,amssymb,pifont,colortbl,amscd, wrapfig, ascmac}
\usepackage{amsthm}
\usepackage{url}
\usepackage{bm}
\newtheorem{theorem}{定理}
\newtheorem{definition}{定義}
\newtheorem{example}{例}

\def\ev{\mathrm{ev}}
\def\c{\mathop{\mathrm{cts}}\nolimits}
\def\odd{\mathop{\mathrm{odd}}\nolimits}
\def\diag{\mathop{\mathrm{diag}}\nolimits}
\def\mod{\mathop{\mathrm{mod}}\nolimits}
\def\deg{\mathop{\mathrm{deg}}\nolimits}
\def\inv{\mathop{\mathrm{Inv}}\nolimits}
\def\cl{\mathop{\mathrm{cl}}\nolimits}
\def\ad{\mathop{\mathrm{ad}}\nolimits}
\newcommand{\sad}{\overline{\ad}}
\def\tr{\mathop{\mathrm{tr}}\nolimits}
\def\End{\mathop{\mathrm{End}}\nolimits}
\def\id{\mathop{\mathrm{id}}\nolimits}
\def\ev{\mathop{\mathrm{ev}}\nolimits}
\def\coev{\mathop{\mathrm{coev}}\nolimits}
\def\coad{\mathop{\mathrm{coad}}\nolimits}
\def\Ob{\mathop{\mathrm{Ob}}\nolimits}
\def\Hom{\mathop{\mathrm{Hom}}\nolimits}
\def\im{\mathop{\mathrm{Im}}\nolimits}
\def\Span{\mathop{\mathrm{Span}}\nolimits}
\def\ideal{\mathop{\mathrm{ideal}}\nolimits}
\def\co{\colon\thinspace}
%for U_h
\newcommand{\uqenh}[1]{ (\bar U_q^{\ev})\,\hat  {}^{\;\hat  \otimes #1}}
\newcommand{\uqen}[1]{ (\bar U_q^{\ev})\;\tilde {}^{\;\tilde \otimes #1}}
\newcommand{\uqe}{\bar U_q^{\ev}}
\newcommand{\uqz}{\bar U_q^0}
\newcommand{\uqze}{\bar U_q^{\ev 0}}
\newcommand{\uq}{\bar U_q}
\newcommand{\muq}{\mathcal{ U}_q}
\newcommand{\muqe}{\mathcal{ U}_q^{\ev}}
\newcommand{\uqzq}{U_{\mathbb{Z},q}}
\newcommand{\uqzqe}{(U_{\mathbb{Z},q})^{\ev}}
\newcommand{\uqn}[1]{\bar U_q^{\otimes {#1}}}
\newcommand{\uhn}[1]{\bar U_h^{\hat \otimes {#1}}}
\newcommand{\f}[1]{\tilde F^{({#1})}}
\newcommand{\e}[1]{\tilde E^{({#1})}}
\newcommand{\Z}{\mathbb{Z}[q,q^{-1}]}
\def\ZA{\mathbb{P}^{\mathrm{fin}}\big( \Hom_{\mathcal{A}}(0,g)\big)}

\def\deg{\mathop{\mathrm{deg}}\nolimits}
\def\inv{\mathop{\mathrm{Inv}}\nolimits}
\def\cl{\mathop{\mathrm{cl}}\nolimits}
\def\ad{\mathop{\mathrm{ad}}\nolimits}
\def\tr{\mathop{\mathrm{tr}}\nolimits}
\def\End{\mathop{\mathrm{End}}\nolimits}
\def\id{\mathop{\mathrm{id}}\nolimits}
\def\ev{\mathop{\mathrm{ev}}\nolimits}
\def\coev{\mathop{\mathrm{coev}}\nolimits}
\def\coad{\mathop{\mathrm{coad}}\nolimits}
\def\Ob{\mathop{\mathrm{Ob}}\nolimits}
\def\Hom{\mathop{\mathrm{Hom}}\nolimits}
\def\Sets{\mathop{\mathrm{Sets}}\nolimits}
\def\im{\mathop{\mathrm{Im}}\nolimits}
\def\co{\colon\thinspace}

%%% ywr extend 
\def\d{\mathrm d}
\def\grad{\mathrm grad}
%%%

\begin{document}

\title{微分積分続論(ベクトル解析)} 
\author{鈴木 咲衣}
\date{平成27年度前期}
\maketitle

\begin{center} {\Large 演習問題4 } \end{center}

\begin{enumerate}
\item \cite[問題2.42]{koba}  次の曲線$\bm r(t)$の$\bm r (t_{0})$での単位接線ベクトルを求めよ.
\begin{enumerate}
\item $\bm r(t)=(t,t^{2})$
\item $\bm r(t)=(\cos t, \sin t)$
\end{enumerate}
\item \cite[章末問題3.5]{koba} 道$L$を曲線$\bm r (t)=(t\cos t, t \sin t)$の$0\leq t\leq 2\pi$の部分とする.
このとき
\begin{enumerate}
\item $f(t)=t$に対して$\int_{L}f ds$を求めよ.
\item $\bm V=(\cos t, 0)$に対して$\int _{L}\bm V \cdot d \bm r$を求めよ.
\end{enumerate}

\item 閉曲線$C$に対して勾配場の$C$上の線積分は常に$0$となることを示せ.

\item \cite[問題6.34]{koba} ベクトル場$\bm V=(y,2x)$は保存力場でないことを示せ.(ヒント:保存力場であれば,定理\ref{kaku}より同じ終点と始点を持つどんな曲線上の線積分も同じ値になる.)


\item \cite[問題6.5]{koba} 次の$U(x,y)$で与えられるスカラー場を等高線表示し,さらに勾配$\mathrm{grad} U$を重ねて図示せよ。
\begin{enumerate}
\item $U(x,y)=x$
\item $U(x,y)=y^{2}$
\item $U(x,y)=x^{2}+y^{2}$
\item $U(x,y)=x^{2}-y^{2}$
\item $U(x,y)=y-x^{2}$
\end{enumerate}
\item \cite[問題3.14]{koba}斜面$(t,\cos t)$を$t=0$から$t=\frac{\pi}{4}$の点まで質量$m$の物質が滑り落ちるときに得る運動エネルギーを,仕事(線積分)を用いて計算せよ.
\end{enumerate}

\newpage

\begin{center} {\Large 演習問題4 解答} \end{center}
  \begin{enumerate}
    \item
      \begin{enumerate}
        \item
          \[\frac {\bm r'(t)}{|\bm r'(t)|} = (\frac{1}{\sqrt{1+4t^2}}, \frac{2t}{\sqrt{1+4t^2}})\]
        \item
          \[\frac {\bm r'(t)}{|\bm r'(t)|} = (-\sin t, \cos t) \]
      \end{enumerate}
    \item
      \begin{enumerate}
        \item
          $\bm r'(t) = (\cos t - t \sin t, \sin t + t \cos t),|\bm r'(t)| = \sqrt{1+t^2}$より、
          \[ \int_L f \d s = \int_0^{2\pi} t \sqrt{1+2t^2} = \left[ \frac{1}{2} \cdot \frac{2}{3}  (1+t^2)^\frac{3}{2} \right]_0^{2\pi} = \frac{1}{3} \left\{ (1+4\pi^2)^\frac{3}{2} - 1 \right\} \]
        \item
          \begin{eqnarray*} 
            \int_L \bm V \cdot \d \bm{r} & = & \int_0^{2\pi} \cos t ( \cos t - t \sin t ) \d t \\
            & = & \int_0^{2\pi} ( \cos^2 t - t \cos t \sin t ) \d t  \\
            & = & \int_0^{2\pi} \left( \frac{\cos 2t + 1}{2} - t \frac{\sin 2t}{2} \right) \d t \\
            & = & \pi + \frac{1}{2} \left( \left[ \frac{t \cos 2t}{2} \right]_0^{2\pi} - \int_0^{2\pi} \frac{\cos 2t}{2} \d t \right)\\
            & = &  \pi + \frac{1}{2} \cdot \pi = \frac{3}{2} \pi 
          \end{eqnarray*}
      \end{enumerate}
    \item
      スカラー場を$U(\bm r)$, 曲線$\bm C: \bm r(t)$ ただし$t1 \leq t \leq t2$とおく。
      \begin{eqnarray*}
        \int_{\bm C} \grad U \cdot \d \bm{r} & = & \int_{\bm r(t1)}^{\bm r(t2)} \frac{\partial U}{\partial x} \d x + \frac{\partial U}{\partial y} \d y + \frac{\partial U}{\partial z} \d z \\
        & = & \int_{t1}^{t2} \left( \frac{\partial U}{\partial x} \frac{\d x}{\d t} + \frac{\partial U}{\partial y} \frac{\d y}{\d t} + \frac{\partial U}{\partial z} \frac{\d z}{\d t} \right) \d t \\
        & = & \int_{t1}^{t2} \frac{\d U}{\d t} \d t = U(\bm r(t2) - U(\bm r(t1))
      \end{eqnarray*}
      より、勾配場の線積分はスカラー場の終点と始点の値の差になることがわかる。よって閉曲線であれば始点と終点が一致するため勾配場の線積分はゼロである。
    \item
      半径$r$の円周上で左回りに$\bm V$を積分する。
      \begin{eqnarray*} 
        \int_{\bm C} \bm V \cdot \d \bm r & = & \int_0^{2\pi} \d \theta ( - \sin^2 \theta + 2 \cos^2 \theta ) \\
        & = & \int_0^{2\pi} \d \theta ( \cos 2\theta + \cos^2 \theta) \\
        & = & \int_0^{2\pi} \d \theta \left( \cos 2\theta + \frac{1+\cos 2\theta}{2} \right) \\
        & = & \pi
      \end{eqnarray*}
      $\bm V$が保存場だと仮定すると、任意の周回積分が消えるはずだがそうなってはいない。よって保存場ではない。
    \item
      \begin{enumerate}
        \item
        \item
        \item
        \item
        \item
        \item
      \end{enumerate}
    \item
      
  \end{enumerate}


\newpage

\begin{thebibliography}{99}
\bibitem{koba} 小林亮,高橋大輔「ベクトル解析入門」(東京大学出版会)
\end{thebibliography}

\end{document}

