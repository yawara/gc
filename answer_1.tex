\documentclass{jarticle}

\setlength{\voffset}{-65pt}
\setlength{\oddsidemargin}{-5mm}
\setlength{\textwidth}{490pt}
\setlength{\textheight}{700pt}

\usepackage{graphicx}
\usepackage{amsmath,amssymb,pifont,colortbl,amscd, wrapfig, ascmac}
\usepackage{amsthm}
\usepackage{url}
\usepackage{bm}
\newtheorem{theorem}{定理}
\newtheorem{definition}{定義}
\newtheorem{example}{例}

\def\ev{\mathrm{ev}}
\def\c{\mathop{\mathrm{cts}}\nolimits}
\def\odd{\mathop{\mathrm{odd}}\nolimits}
\def\diag{\mathop{\mathrm{diag}}\nolimits}
\def\mod{\mathop{\mathrm{mod}}\nolimits}
\def\deg{\mathop{\mathrm{deg}}\nolimits}
\def\inv{\mathop{\mathrm{Inv}}\nolimits}
\def\cl{\mathop{\mathrm{cl}}\nolimits}
\def\ad{\mathop{\mathrm{ad}}\nolimits}
\newcommand{\sad}{\overline{\ad}}
\def\tr{\mathop{\mathrm{tr}}\nolimits}
\def\End{\mathop{\mathrm{End}}\nolimits}
\def\id{\mathop{\mathrm{id}}\nolimits}
\def\ev{\mathop{\mathrm{ev}}\nolimits}
\def\coev{\mathop{\mathrm{coev}}\nolimits}
\def\coad{\mathop{\mathrm{coad}}\nolimits}
\def\Ob{\mathop{\mathrm{Ob}}\nolimits}
\def\Hom{\mathop{\mathrm{Hom}}\nolimits}
\def\im{\mathop{\mathrm{Im}}\nolimits}
\def\Span{\mathop{\mathrm{Span}}\nolimits}
\def\ideal{\mathop{\mathrm{ideal}}\nolimits}
\def\co{\colon\thinspace}
%for U_h
\newcommand{\uqenh}[1]{ (\bar U_q^{\ev})\,\hat  {}^{\;\hat  \otimes #1}}
\newcommand{\uqen}[1]{ (\bar U_q^{\ev})\;\tilde {}^{\;\tilde \otimes #1}}
\newcommand{\uqe}{\bar U_q^{\ev}}
\newcommand{\uqz}{\bar U_q^0}
\newcommand{\uqze}{\bar U_q^{\ev 0}}
\newcommand{\uq}{\bar U_q}
\newcommand{\muq}{\mathcal{ U}_q}
\newcommand{\muqe}{\mathcal{ U}_q^{\ev}}
\newcommand{\uqzq}{U_{\mathbb{Z},q}}
\newcommand{\uqzqe}{(U_{\mathbb{Z},q})^{\ev}}
\newcommand{\uqn}[1]{\bar U_q^{\otimes {#1}}}
\newcommand{\uhn}[1]{\bar U_h^{\hat \otimes {#1}}}
\newcommand{\f}[1]{\tilde F^{({#1})}}
\newcommand{\e}[1]{\tilde E^{({#1})}}
\newcommand{\Z}{\mathbb{Z}[q,q^{-1}]}
\def\ZA{\mathbb{P}^{\mathrm{fin}}\big( \Hom_{\mathcal{A}}(0,g)\big)}

\def\deg{\mathop{\mathrm{deg}}\nolimits}
\def\inv{\mathop{\mathrm{Inv}}\nolimits}
\def\cl{\mathop{\mathrm{cl}}\nolimits}
\def\ad{\mathop{\mathrm{ad}}\nolimits}
\def\tr{\mathop{\mathrm{tr}}\nolimits}
\def\End{\mathop{\mathrm{End}}\nolimits}
\def\id{\mathop{\mathrm{id}}\nolimits}
\def\ev{\mathop{\mathrm{ev}}\nolimits}
\def\coev{\mathop{\mathrm{coev}}\nolimits}
\def\coad{\mathop{\mathrm{coad}}\nolimits}
\def\Ob{\mathop{\mathrm{Ob}}\nolimits}
\def\Hom{\mathop{\mathrm{Hom}}\nolimits}
\def\Sets{\mathop{\mathrm{Sets}}\nolimits}
\def\im{\mathop{\mathrm{Im}}\nolimits}
\def\co{\colon\thinspace}


\begin{document}

\title{微分積分続論(ベクトル解析)} 
\author{鈴木 咲衣}
\date{平成27年度前期}
\maketitle

\begin{center} {\Large 演習問題1 } \end{center}

\begin{enumerate}
\item 次のベクトル$\bm u, \bm v$に対して内積$\bm u \cdot \bm v$を求めよ.また,関係式(\ref{cos})$\bm u \cdot \bm v=|\bm u| |\bm v|\cos \theta $を用いて,$\bm u, \bm v$の間のなす各 $\theta$($0\leq \theta \leq \pi$)を求めよ.
\begin{enumerate}
\item $\bm u= (-1, 0, 7), \bm v= (-4, 5,3)$ \label{a}
\item $\bm u= (-1, 1, 0), \bm v= (1, -2, 2)$ \label{b}
\end{enumerate}

\item 3次元ベクトル$\bm u\not =\bm 0, \bm v\not= \bm 0$に対して「$\bm u$と$\bm v$が直行 $\iff$ $\bm u \cdot \bm v=0$」を示せ.
\item 次のベクトル$\bm u, \bm v$に対して外積$\bm u \times \bm v$を求めよ.
\begin{enumerate}
\item $\bm u= (-2, 3, 7), \bm v= (-4, 2,3)$
\item $\bm u= (4, 1, 1), \bm v= (0, -2, 2)$
\end{enumerate}

\item 次の主張を確かめよ.「$3$次元ベクトル$\bm u$と$\bm v$に対して,外積$\bm u \times \bm v$は$3$次元ベクトルで,矢印の向きは$\bm u$から$\bm v$方向に右ネジを回してネジが進む方向、大きさは$\bm u$と$\bm v$の張る平行四辺形の面積である.」
\end{enumerate}

\newpage

\begin{center} {\Large 演習問題1 解答} \end{center}

\begin{enumerate}
  \item
    \begin{enumerate}
      \item 
        \[ \bm u \cdot \bm v = (-1) \times (-4) + 0 \times 5 + 7 \times 3 = 25 \]
        \[ \cos \theta = \frac { \bm u \cdot \bm v } { \bm u \bm v } = \frac { 25 } { \sqrt{50} \sqrt{50} } = \frac{1}{2} \]
        \[ \theta = \frac{\pi}{3} \]
      \item
        \[ \bm u \cdot \bm v = (-1) \times 1 + 1 \times (-2) + 0 \times 2 = -3\] 
        \[ \cos \theta = \frac { \bm u \cdot \bm v } { \bm u \bm v } = \frac { -3 } { \sqrt 2 \sqrt 9 } = - \frac { 1 } { \sqrt 2 } \]
        \[ \theta = \frac{3}{4} \pi \]
    \end{enumerate}
  \item
    $\bm u \neq \bm 0, \bm v \neq \bm 0$ すなわち$|\bm u| \neq 0, |\bm v| \neq 0$ また $\bm u \cdot \bm v = |\bm u| |\bm v| \cos \theta$ より、
  \[ \bm u \perp \bm v \iff \cos \theta = 0 \iff \bm u \cdot \bm v = |\bm u| |\bm v| \cos \theta = 0 \]
  \item
    \begin{enumerate}
      \item $\bm u \times \bm v = (-5,-22,8)$
      \item $\bm u \times \bm v = (4,-8,-8)$
    \end{enumerate}
  
  \item
    $\bm u = (a,b,c)$と$\bm v=(x,y,z)$とおく。また$\bm u$と$\bm v$のなす角を$\theta$とおく。
    
    $\bm u \times \bm v = (bz-cy,cx-az,ay-bx)$ より、
    \[ \bm u \cdot (\bm u \times \bm v) = a(bz-cy)+b(xz-az)+c(ay-bx) = 0\]
    また
    \[ \bm v \cdot (\bm u \times \bm v) = x(bz-cy)+y(xz-az)+z(ay-bx) = 0\]
    よって$\bm u \times \bm v$は$\bm u$と$\bm v$に直行している。すなわち$\bm u$と$\bm v$の張る平面に直行していることがわかる。
    \begin{eqnarray*}
      |\bm u \times \bm v|^2 & = & (bz-cy)^2+(cx-az)^2+(ay-bx)^2 \\
      & = & (a^2+b^2+c^2)(x^2+y^2+z^2) - (ax+by+cz)^2 \\
      & = & |\bm u|^2 |\bm v|^2 - (\bm u \cdot \bm v)^2 \\ 
      & = & |\bm u|^2 |\bm v|^2 - |\bm u|^2 |\bm v|^2 \cos^2 \theta \\
      & = & |\bm u|^2 |\bm v|^2 ( 1 - \cos^2 \theta) \\
      & = & |\bm u|^2 |\bm v|^2 \sin^2 \theta
    \end{eqnarray*}
    より$|\bm u \times \bm v| = |\bm u||\bm v| |\sin \theta|$である。
    よって$\bm u \times \bm v$の大きさは$\bm u$と$\bm v$の張る平行四辺形の面積に等しい。
    
    $\left[ \bm u, \bm v, \bm u \times \bm v \right]$の行列式が正であることが言えれば、$\bm u \times \bm v$の矢印の向きは$\bm u$から$\bm v$方向に右ネジを回してネジが進む方向ということが言える。
    \begin{eqnarray*} 
      \left| 
        \begin{array}{ccc}
          a & b & c \\
          x & y & z \\
          bz-cy & cx-az & ay-bx 
        \end{array}
      \right| 
    & = & a^{2} y^{2} + a^{2} z^{2} - 2 a b x y - 2 a c x z + b^{2} x^{2} + b^{2} z^{2} - 2 b c y z + c^{2} x^{2} + c^{2} y^{2} \\
    & = & (bz-cy)^2+(cx-az)^2+(ay-bx)^2 > 0
    \end{eqnarray*}
\end{enumerate}

\end{document}